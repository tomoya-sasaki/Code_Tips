%\documentclass[handout]{beamer}
\documentclass[11pt]{beamer}          

\mode<presentation>
{
 \setbeamercovered{invisible}
 \usetheme{Madrid}
 % \usetheme{Darmstadt} \useoutertheme[subsection=false]{miniframes}
 % \usecolortheme{beaver}  % if you turn it on, it is red-gray
}

\makeatletter
\newcommand*{\overlaynumber}{\number\beamer@minimum}
\makeatother

\usepackage{verbatim}
\usepackage{graphicx}
\usepackage{amsmath}
\usepackage{bbm} % for a indicator function
\usepackage{multirow,multicol}
\usepackage{latexsym}
\usepackage{colortbl,xcolor}
\usepackage{appendixnumberbeamer}
\usepackage{pifont}
\usepackage{kantlipsum}
\usepackage[utf8]{inputenc}
\graphicspath{{./figure/}{../paper/figure/}}

% === dcolumn package ===
\usepackage{dcolumn}
\newcolumntype{.}{D{.}{.}{-1}}
\newcolumntype{d}[1]{D{.}{.}{#1}}
\newcommand{\bX}{\mathbf{X}}
\newcommand{\vname}[1]{{\color{magenta}{\texttt{#1}}}}
\newcommand{\bone}{\mathbf{1}}

% === Title Page ===
\setbeamertemplate{title page}[default][colsep=-4bp,rounded=true]  % no shdow

% == bottom bar ===
\setbeamertemplate{footline}[frame number]  % only show the number

% === Blocks (block, exampleblock, aleartblock) ===
\setbeamertemplate{blocks}[rounded][shadow=false]  % no shadow

% === Cross out sentences === 
\usepackage[normalem]{ulem}
\newcommand\redsout{\bgroup\markoverwith{\textcolor{red}{\rule[0.5ex]{6pt}{1.5pt}}}\ULon}

% === Multirow cells ===
\usepackage{multirow, booktabs}

% ==== dotted lines in tables ===
\usepackage{arydshln}
\usepackage{threeparttable}

% === appendix numbers ===
\usepackage{appendixnumberbeamer}

% === table ===
\usepackage{array}
\usepackage{caption}

% === button ===
\usepackage{tikz}
\setbeamertemplate{navigation symbols}{}
\setbeamertemplate{itemize items}[circle]
\setbeamertemplate{enumerate items}[circle]

% === checkmark (pifont) ===
\newcommand{\cmark}{\ding{52}}%
\newcommand{\xmark}{\ding{55}}%


% === begin document

\begin{document}

\newcommand{\indep}{\stackrel{ \text{indep.}}{\sim}}
\def\independenT#1#2{\mathrel{\rlap{$#1#2$}\mkern2mu{#1#2}}}
% === new commands ===

% new colors
\definecolor{light-gray}{gray}{0.9}

\newcommand{\difp}{\hspace{-0.15in}$\rangle$}

\newcolumntype{L}[1]{>{\raggedright\let\newline\\\arraybackslash\hspace{0pt}}m{#1}}
\newcolumntype{C}[1]{>{\centering\let\newline\\\arraybackslash\hspace{0pt}}m{#1}}
\newcolumntype{R}[1]{>{\raggedleft\let\newline\\\arraybackslash\hspace{0pt}}m{#1}}


\newcommand\spacingset[1]{\renewcommand{\baselinestretch}%
{#1}\small\normalsize}

% \beamerdefaultoverlayspecification{<+->}  % for presentation
\usenavigationsymbolstemplate{}  % no navigation bar
% \beamerdefaultoverlayspecification{<+->}  % automatic pause

%-- to show the bar, please comment out `footline` option
\title[Title Short]{Title}
\author[Author names in the bottom]{{\normalsize
\begin{tabular}[h]{cc}
  Author 1 &  Author 2 \\[2.5mm]
  Institution 1 & Institution 2
\end{tabular}
}}
% \institute[Institute]{{\normalsize Institute 1 \hspace{.75in} Institute 2}}
\date[Date]{\medskip Talk at the Workshop \\ Place \\ \bigskip Month xx,  20xx}


\section{Title Page}
\begin{frame}
\titlepage
\end{frame}

\section{Introduction}
\begin{frame}{Motivation and Overview}
\begin{itemize}
  \item Text 
  \vfill
  \item Text \hfill Text
\end{itemize}
\begin{alertblock}{Show 1}
A thing to show,
\end{alertblock}
\begin{exampleblock}{Show 2}
A thing to show 2.
\end{exampleblock}
\end{frame}


\end{document}
