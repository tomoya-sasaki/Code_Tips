\documentclass[a4paper,10.5pt]{jsarticle}  %jsarticleでも良いかも
%--脚注の設定
\usepackage{natbib}
\bibpunct[, ]{(}{)}{;}{and}{}{,} %本文での引用の体裁はここで整えられるみたい
\bibliographystyle{apsr2006-2}   %このagsmは編集済みなので注意。ジャーナルが太字にならないようにしてる。ブログ参照。
\usepackage{amsmath}
%--余白の設定   http://pcwide-jp.blogspot.co.uk/2009/07/latex.html  より
\setlength{\topmargin}{20mm}
\addtolength{\topmargin}{-1in}
\setlength{\oddsidemargin}{20mm}
\addtolength{\oddsidemargin}{-1in}
\setlength{\evensidemargin}{15mm}
\addtolength{\evensidemargin}{-1in}
\setlength{\textwidth}{170mm}
\setlength{\textheight}{254mm}
\setlength{\headsep}{0mm}
\setlength{\headheight}{0mm}
\setlength{\topskip}{0mm}
%--図の設定
\usepackage[dvipdfmx]{graphicx} % PDFの利用もOKに
  %\usepackage[dvips]{graphicx}
%--行間の設定
\usepackage{setspace} 
\setstretch{1.13} % ページ全体の行間を設定
%コードの設定
\usepackage{listings}
\usepackage{color}
\definecolor{dkgreen}{rgb}{0,0.6,0}
\definecolor{mygray}{rgb}{0.5,0.5,0.5}
\definecolor{mauve}{rgb}{0.58,0,0.82}

\definecolor{codegreen}{rgb}{0,0.6,0}
\definecolor{codegray}{rgb}{0.5,0.5,0.5}
\definecolor{codepurple}{rgb}{0.58,0,0.82}
\definecolor{backgroundcolour}{rgb}{0.95,0.95,0.92}

\lstset{ %
  language= Python, %ここを含めなければ、変に太字になったりしない
  aboveskip=2.5mm,
  belowskip=4.5mm,
  showstringspaces=false,
  columns=flexible,
  keepspaces=true,
  numbers=left,                    
  numbersep=5pt,    
  basicstyle={\small\ttfamily},
  commentstyle={\small\ttfamily},
  breaklines=true,
  breakatwhitespace=true
  tabsize=3
  backgroundcolor=\color{backgroundcolour},   
  commentstyle=\color{codegreen},
  keywordstyle=\color{magenta},
  numberstyle=\tiny\color{codegray},
  stringstyle=\color{codepurple},
	xleftmargin = 1.1cm,
  framexleftmargin = 1em
}
\usepackage{xcolor}
\usepackage{framed}
\colorlet{shadecolor}{green!8}
%リンクの埋め込みを可能にする  \href{ **URL** }{表示テキスト}
\usepackage[dvipdfmx]{hyperref}
\usepackage{courier}
\usepackage{here}
%日本語環境用の設定を追加
\renewcommand{\refname}{参考文献}
\renewcommand{\abstractname}{要約}
\renewcommand{\contentsname}{Contents}
\usepackage{multirow}
\usepackage{placeins}
%数学用のフォント
\usepackage{amsmath} 
\usepackage{amssymb}
\usepackage{amsfonts} 
%-----------------
\begin{document}

\title{Probability Theory}
\author{}
\date{}
\maketitle
%\begin{abstract}
%\end{abstract}

\setcounter{tocdepth}{1}
\tableofcontents

\section{Bayes' Theorem and Independence}
From \href{http://math.stackexchange.com/questions/1897874/bayes-theorem-and-independence}{StackExchange}.
\subsection{Question}
I'm deriving Probabilistic latent semantic analysis model. In the model, documents $d$ and words $w$ are observed. 
\begin{align*} \Pr(d,w) &= \sum_{c} \Pr (d,w,c) \\ &= \sum_{c} \Pr(d) \Pr(w,c|d)\\ &= \Pr(d) \sum_{c} \frac{\Pr(w,c,d)}{\Pr(d)} \\ &= \Pr(d) \sum_{c} \Pr(w|c,d) \Pr(c|d)  \end{align*} 

I used Bayes' theorem from the second line to the third. However, Wikipedia says $\Pr(d) \sum_{c} \Pr(w|c) \Pr(c|d)$. Is this because $d$ and $w$ are independent? If so, how can I know the independence from the grahical model?

\subsection{Answer 1}
From what Wikipedia says, $d$ and $w$ are assumed to be conditionally independent given $c$. This means that $$\Pr(d,w,c) = \Pr(c)\Pr(d|c)\Pr(w|c).$$ Now using Bayes's theorem we obtain that this quantity equals $\Pr(d)\Pr(c|d)\Pr(w|c)$.

You could also continue your line of thought, because conditional independence is equivalent to saying that $\Pr(w|c,d) = \Pr(w|c)$.

\subsection{Answer 2}
Immediately skip to the last line by just applying the definition of conditional probability.

\begin{align*} \Pr(d,w) &= \sum_{c} \Pr (d,w,c) \\ &= \sum_{c} \Pr(d) \Pr(w,c\mid d)\\ &= \Pr(d) \sum_{c} \Pr(w\mid c,d) \Pr(c\mid d) \end{align*}
Anyhow, to say that $\Pr(w\mid c,d)=\Pr(w\mid c)$ is to assert that $w,d$ are conditionally independent for any given $c$. Conditional Independence of $w,d$ given $c$ is defined as when: $\Pr(w,d\mid c)=\Pr(w\mid c)\Pr(d\mid c)$

So we have:
\begin{align*} \Pr(d,w) &= \sum_{c} \Pr (d,w,c) && \text{Law of Total Probability} \\ &= \sum_{c} \Pr(c) \Pr(d,w\mid c) && \text{defn. Conditional Probability}\\ &= \sum_{c} \Pr(c) \Pr(d\mid c) \Pr(w\mid c) && \text{because conditional independence} \\ &= \sum_c \Pr(c\mid d)\Pr(d)\Pr(w\mid c) && \text{Bayes' Rule}\\ &= \Pr(d)\sum_c \Pr(w\mid c)\Pr(c\mid d) 
\end{align*}
\end{document}
